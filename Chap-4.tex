% !Tex program = pdflatex
% Chapter 4 - Quantum circuit
\ifx\allfiles\undefined
\documentclass[en]{sol-man}
\begin{document}
\fi
\chapter{Quantum circuit}

\section{Quantum algorithm}

\section{Single qubit operations}

\begin{exe}
    In Exercise 2.11, which you should do now if you haven't already done it, you compute the eigenvectors of the Pauli matrices. Find the points on the Bloch sphere which corresponding to the normalized eigenvectors of the different Pauli matrices.
\end{exe}
\begin{sol}
    The normalized eigenvectors of $X$ are
    \begin{align}
        \frac{1}{\sqrt{2}}(\lvert 0\rangle+\lvert 1\rangle),\quad\frac{1}{\sqrt{2}}(\lvert 0\rangle-\lvert 1\rangle),
    \end{align}
    which corresponds to the intersection points of the Bloch sphere and the positive $x$ axis, and of the Bloch sphere and the negative $x$ axis.\\
    The normalized eigenvectors of $Y$ are
    \begin{align}
        \frac{1}{\sqrt{2}}(\lvert 0\rangle+i\lvert 1\rangle),\quad\frac{1}{\sqrt{2}}(\lvert 0\rangle-i\lvert 1\rangle),
    \end{align}
    which corresponds to the intersection points of the Bloch sphere and the positive $y$ axis, and of the Bloch sphere and the negative $y$ axis.\\
    The normalized eigenvectors of $Z$ are
    \begin{align}
        \lvert 0\rangle,\quad\lvert 1\rangle,
    \end{align}
    which corresponds to the intersection points of the Bloch sphere and the positive $z$ axis, and of the Bloch sphere and the negative $z$ axis.
\end{sol}

\begin{exe}
    Let $x$ be a real number and $A$ a matrix such that $A^2=I$. Show that
    \begin{align}
        \label{E4.2}
        \exp(iAx)=\cos(x)I+i\sin(x)A.
    \end{align}
    Use this result to verify Equation (4.4) through (4.6)\footnote{Equation 4.4: $R_x(\theta)\equiv e^{-i\theta X/2}=\cos\frac{\theta}{2}I-i\sin\frac{\theta}{2}X=\left[\begin{matrix}
        \cos\frac{\theta}{2}&-i\sin\frac{\theta}{2}\\
        -i\sin\frac{\theta}{2}&\cos\frac{\theta}{2}
    \end{matrix}\right]$;\\
    Equation 4.5: $R_y(\theta)\equiv e^{-i\theta Y/2}=\cos\frac{\theta}{2}I-i\sin\frac{\theta}{2}Y=\left[\begin{matrix}
        \cos\frac{\theta}{2}&-\sin\frac{\theta}{2}\\
        \sin\frac{\theta}{2}&\cos\frac{\theta}{2}
    \end{matrix}\right]$;\\
    Equation 4.6: $R_z(\theta)\equiv e^{-i\theta Z/2}=\cos\frac{\theta}{2}I-i\sin\frac{\theta}{2}Z=\left[\begin{matrix}
        e^{-i\theta/2}&0\\
        0&e^{i\theta/2}
    \end{matrix}\right]$.}.
\end{exe}
\begin{pf}
    Similar to Exercise 2.35 and Problem 2.1, the left side of Equation \eqref{E4.2} is
    \begin{align}
        \exp(iAx)=\sum_{n=0}^{\infty}\frac{1}{n!}(iAx)^n=\sum_{n=1}^{\infty}\frac{(-1)^n}{(2n)!}(Ax)^{2n}+\sum_{n=0}^{\infty}\frac{i(-1)^n}{(2n+1)!}(Ax)^{2n+1}.
    \end{align}
    Note that
    \begin{align}
        A^2=I,
    \end{align}
    so
    \begin{align}
        \exp(iAx)=\sum_{n=0}^{\infty}\frac{(-1)^n}{(2n)!}+\sum_{n=0}^{\infty}\frac{i(-1)^n}{(2n+1)!}Ax^{2n+1}=\cos(x)I+i\sin(x)A,
    \end{align}
    which equals the right side of Equation \eqref{E4.2}.
    Therefore, Equation \eqref{E4.2} holds.

    Using the above result,
    \begin{align}
        \notag R_x(\theta)=&e^{-i\theta X/2}=\cos\left(-\frac{\theta}{2}\right)I+i\sin\left(-\frac{\theta}{2}\right)X=\cos\frac{\theta}{2}I-i\sin\frac{\theta}{2}X=\cos\frac{\theta}{2}\left[\begin{matrix}
            1&0\\
            0&1
        \end{matrix}\right]-i\sin\frac{\theta}{2}\left[\begin{matrix}
            0&1\\
            1&0
        \end{matrix}\right]\\
        =&\left[\begin{matrix}
            \cos\frac{\theta}{2}&-i\sin\frac{\theta}{2}\\
            -i\sin\frac{\theta}{2}&\cos\frac{\theta}{2}
        \end{matrix}\right],
    \end{align}
    which verifies Equation (4.4),
    \begin{align}
        \notag R_Y(\theta)=&e^{-i\theta Y/2}=\cos\left(-\frac{\theta}{2}\right)I+i\sin\left(-\frac{\theta}{2}\right)Y=\cos\frac{\theta}{2}-i\sin\frac{\theta}{2}Y=\cos\frac{\theta}{2}\left[\begin{matrix}
            1&0\\
            0&1
        \end{matrix}\right]-i\sin\frac{\theta}{2}\left[\begin{matrix}
            0&-i\\
            i&0
        \end{matrix}\right]\\
        =&\left[\begin{matrix}
            \cos\frac{\theta}{2}&-\sin\frac{\theta}{2}\\
            \sin\frac{\theta}{2}&\cos\frac{\theta}{2}
        \end{matrix}\right],
    \end{align}
    which verifies Equation (4.5), and
    \begin{align}
        \notag R_Z(\theta)=&e^{-i\theta Z/2}=\cos\left(-\frac{\theta}{2}\right)I+i\sin\left(-\frac{\theta}{2}\right)Z=\cos\frac{\theta}{2}-i\sin\frac{\theta}{2}Z=\cos\frac{\theta}{2}\left[\begin{matrix}
            1&0\\
            0&1
        \end{matrix}\right]-i\sin\frac{\theta}{2}\left[\begin{matrix}
            1&0\\
            0&-1
        \end{matrix}\right]\\
        =&\left[\begin{matrix}
            \cos\frac{\theta}{2}-i\sin\frac{\theta}{2}&0\\
            0&\cos\frac{\theta}{2}+i\sin\frac{\theta}{2}
        \end{matrix}\right]=\left[\begin{matrix}
            e^{-i\theta/2}&0\\
            0&e^{i\theta/2}
        \end{matrix}\right],
    \end{align}
    which verifies Equation (4.6).
\end{pf}

\begin{exe}
    Show that, up to a global phase, the $\pi/8$ gate satisfies $T=R_z(\pi/4)$.
\end{exe}
\begin{pf}
    Using the conclusion proved in the previous exercise,
    \begin{align}
        R_z(\pi/4)=\left[\begin{matrix}
            e^{-i\pi/8}&0\\
            0&e^{i\pi/8}
        \end{matrix}\right]=\exp(-i\pi/8)T,
    \end{align}
    so the $\pi/8$ gate satisfies $T=R_z(\pi/4)$ up to a global phase.
\end{pf}

\ifx\allfiles\undefined
\end{document}
\fi