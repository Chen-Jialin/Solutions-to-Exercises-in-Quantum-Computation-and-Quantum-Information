% !Tex program = pdflatex
% Chapter 4 - Quantum circuit
\ifx\allfiles\undefined
\documentclass[en]{sol-man}
\begin{document}
\fi
\chapter{Quantum circuit}

\section{Quantum algorithm}

\section{Single qubit operations}

\begin{exe}
    In Exercise 2.11, which you should do now if you haven't already done it, you compute the eigenvectors of the Pauli matrices. Find the points on the Bloch sphere which corresponding to the normalized eigenvectors of the different Pauli matrices.
\end{exe}
\begin{sol}
    The normalized eigenvectors of $X$ are
    \begin{align}
        \frac{1}{\sqrt{2}}(\lvert 0\rangle+\lvert 1\rangle),\quad\frac{1}{\sqrt{2}}(\lvert 0\rangle-\lvert 1\rangle),
    \end{align}
    which corresponds to the intersection points of the Bloch sphere and the positive $x$ axis, and of the Bloch sphere and the negative $x$ axis.\\
    The normalized eigenvectors of $Y$ are
    \begin{align}
        \frac{1}{\sqrt{2}}(\lvert 0\rangle+i\lvert 1\rangle),\quad\frac{1}{\sqrt{2}}(\lvert 0\rangle-i\lvert 1\rangle),
    \end{align}
    which corresponds to the intersection points of the Bloch sphere and the positive $y$ axis, and of the Bloch sphere and the negative $y$ axis.\\
    The normalized eigenvectors of $Z$ are
    \begin{align}
        \lvert 0\rangle,\quad\lvert 1\rangle,
    \end{align}
    which corresponds to the intersection points of the Bloch sphere and the positive $z$ axis, and of the Bloch sphere and the negative $z$ axis.
\end{sol}

\begin{exe}
    Let $x$ be a real number and $A$ a matrix such that $A^2=I$. Show that
    \begin{align}
        \label{E4.2}
        \exp(iAx)=\cos(x)I+i\sin(x)A.
    \end{align}
    Use this result to verify Equation (4.4) through (4.6)\footnote{Equation 4.4: $R_x(\theta)\equiv e^{-i\theta X/2}=\cos\frac{\theta}{2}I-i\sin\frac{\theta}{2}X=\left[\begin{matrix}
        \cos\frac{\theta}{2}&-i\sin\frac{\theta}{2}\\
        -i\sin\frac{\theta}{2}&\cos\frac{\theta}{2}
    \end{matrix}\right]$;\\
    Equation 4.5: $R_y(\theta)\equiv e^{-i\theta Y/2}=\cos\frac{\theta}{2}I-i\sin\frac{\theta}{2}Y=\left[\begin{matrix}
        \cos\frac{\theta}{2}&-\sin\frac{\theta}{2}\\
        \sin\frac{\theta}{2}&\cos\frac{\theta}{2}
    \end{matrix}\right]$;\\
    Equation 4.6: $R_z(\theta)\equiv e^{-i\theta Z/2}=\cos\frac{\theta}{2}I-i\sin\frac{\theta}{2}Z=\left[\begin{matrix}
        e^{-i\theta/2}&0\\
        0&e^{i\theta/2}
    \end{matrix}\right]$.}.
\end{exe}
\begin{pf}
    Similar to Exercise 2.35 and Problem 2.1, the left side of Equation \eqref{E4.2} is
    \begin{align}
        \exp(iAx)=\sum_{n=0}^{\infty}\frac{1}{n!}(iAx)^n=\sum_{n=1}^{\infty}\frac{(-1)^n}{(2n)!}(Ax)^{2n}+\sum_{n=0}^{\infty}\frac{i(-1)^n}{(2n+1)!}(Ax)^{2n+1}.
    \end{align}
    Note that
    \begin{align}
        A^2=I,
    \end{align}
    so
    \begin{align}
        \exp(iAx)=\sum_{n=0}^{\infty}\frac{(-1)^n}{(2n)!}+\sum_{n=0}^{\infty}\frac{i(-1)^n}{(2n+1)!}Ax^{2n+1}=\cos(x)I+i\sin(x)A,
    \end{align}
    which equals the right side of Equation \eqref{E4.2}.
    Therefore, Equation \eqref{E4.2} holds.

    Using the above result,
    \begin{align}
        \notag R_x(\theta)=&e^{-i\theta X/2}=\cos\left(-\frac{\theta}{2}\right)I+i\sin\left(-\frac{\theta}{2}\right)X=\cos\frac{\theta}{2}I-i\sin\frac{\theta}{2}X=\cos\frac{\theta}{2}\left[\begin{matrix}
            1&0\\
            0&1
        \end{matrix}\right]-i\sin\frac{\theta}{2}\left[\begin{matrix}
            0&1\\
            1&0
        \end{matrix}\right]\\
        =&\left[\begin{matrix}
            \cos\frac{\theta}{2}&-i\sin\frac{\theta}{2}\\
            -i\sin\frac{\theta}{2}&\cos\frac{\theta}{2}
        \end{matrix}\right],
    \end{align}
    which verifies Equation (4.4),
    \begin{align}
        \notag R_Y(\theta)=&e^{-i\theta Y/2}=\cos\left(-\frac{\theta}{2}\right)I+i\sin\left(-\frac{\theta}{2}\right)Y=\cos\frac{\theta}{2}I-i\sin\frac{\theta}{2}Y=\cos\frac{\theta}{2}\left[\begin{matrix}
            1&0\\
            0&1
        \end{matrix}\right]-i\sin\frac{\theta}{2}\left[\begin{matrix}
            0&-i\\
            i&0
        \end{matrix}\right]\\
        =&\left[\begin{matrix}
            \cos\frac{\theta}{2}&-\sin\frac{\theta}{2}\\
            \sin\frac{\theta}{2}&\cos\frac{\theta}{2}
        \end{matrix}\right],
    \end{align}
    which verifies Equation (4.5), and
    \begin{align}
        \notag R_Z(\theta)=&e^{-i\theta Z/2}=\cos\left(-\frac{\theta}{2}\right)I+i\sin\left(-\frac{\theta}{2}\right)Z=\cos\frac{\theta}{2}I-i\sin\frac{\theta}{2}Z=\cos\frac{\theta}{2}\left[\begin{matrix}
            1&0\\
            0&1
        \end{matrix}\right]-i\sin\frac{\theta}{2}\left[\begin{matrix}
            1&0\\
            0&-1
        \end{matrix}\right]\\
        =&\left[\begin{matrix}
            \cos\frac{\theta}{2}-i\sin\frac{\theta}{2}&0\\
            0&\cos\frac{\theta}{2}+i\sin\frac{\theta}{2}
        \end{matrix}\right]=\left[\begin{matrix}
            e^{-i\theta/2}&0\\
            0&e^{i\theta/2}
        \end{matrix}\right],
    \end{align}
    which verifies Equation (4.6).
\end{pf}

\begin{exe}
    Show that, up to a global phase, the $\pi/8$ gate satisfies $T=R_z(\pi/4)$.
\end{exe}
\begin{pf}
    Using the conclusion proved in the previous exercise,
    \begin{align}
        R_z(\pi/4)=\left[\begin{matrix}
            e^{-i\pi/8}&0\\
            0&e^{i\pi/8}
        \end{matrix}\right]=\exp(-i\pi/8)T,
    \end{align}
    so the $\pi/8$ gate satisfies $T=R_z(\pi/4)$ up to a global phase.
\end{pf}

\begin{exe}
    Express the Hadamard gate $H$ as a product of $R_x$ and $R_z$ rotations and $e^{i\varphi}$ for some $\varphi$.
\end{exe}
\begin{pf}
    Since
    \begin{align}
        \notag e^{i\pi/2}R_z\left(\frac{\pi}{2}\right)R_x\left(\frac{\pi}{2}\right)R_z\left(\frac{\pi}{2}\right)=&e^{i\pi/2}\left[\begin{matrix}
            e^{-i\pi/4}&0\\
            0&e^{i\pi/4}
        \end{matrix}\right]\left[\begin{matrix}
            \cos\frac{\pi}{4}&-i\sin\frac{\pi}{4}\\
            -i\sin\frac{\pi}{4}&\cos\frac{\pi}{4}
        \end{matrix}\right]\left[\begin{matrix}
            e^{-i\pi/4}&0\\
            0&e^{i\pi/4}
        \end{matrix}\right]\\
        \notag=&e^{i\pi/2}\left[\begin{matrix}
            e^{-i\pi/4}\cos\frac{\pi}{4}&-ie^{-i\pi/4}\sin\frac{\pi}{4}\\
            -ie^{i\pi/4}\sin\frac{\pi}{4}&e^{i\pi/4}\cos\frac{\pi}{4}
        \end{matrix}\right]\left[\begin{matrix}
            e^{-i\pi/4}&0\\
            0&e^{i\pi/4}
        \end{matrix}\right]\\
        \notag=&e^{i\pi/2}\left[\begin{matrix}
            e^{-i\pi/2}\cos\frac{\pi}{4}&-i\sin\frac{\pi}{4}\\
            -i\sin\frac{\pi}{4}&e^{i\pi/2}\cos\frac{\pi}{4}
        \end{matrix}\right]=\frac{1}{\sqrt{2}}\left[\begin{matrix}
            1&1\\
            1&-1
        \end{matrix}\right]=H,
    \end{align}
    the Hadamard gate $H$ can be expressed as a product of $R_x$ and $R_z$ rotations and $e^{i\varphi}$.
\end{pf}

\begin{exe}
    Prove that $(\hat{n}\cdot\vec{\sigma})^2=I$, and use this to verify Equation (4.8)\footnote{$R_{\hat{n}}(\theta)\equiv\exp(-i\theta\hat{n}\cdot\vec{\sigma}/2)=\cos\left(\frac{\theta}{2}\right)I-i\sin\left(\frac{\theta}{2}\right)(n_xX+n_yY+n_zZ)$.}.
\end{exe}
\begin{pf}
    Similar to Exercise 2.35,
    \begin{gather}
        \hat{n}\cdot\vec{\sigma}=n_x\sigma_1+n_y\sigma_2+n_z\sigma_3=n_x\left[\begin{matrix}
            0&1\\
            1&0
        \end{matrix}\right]+n_y\left[\begin{matrix}
            0&-i\\
            i&0
        \end{matrix}\right]+n_z\left[\begin{matrix}
            1&0\\
            0&-1
        \end{matrix}\right]=\left[\begin{matrix}
            n_z&n_x-in_y\\
            n_x+in_y&-n_z
        \end{matrix}\right],\\
        \Longrightarrow(\hat{n}\cdot\vec{\sigma})^2=\left[\begin{matrix}
            n_z&n_x-in_y\\
            n_x+in_y&-n_z
        \end{matrix}\right]\left[\begin{matrix}
            n_z&n_x-in_y\\
            n_x+in_y&-n_z
        \end{matrix}\right]=\left[\begin{matrix}
            n_x^2+n_y^2+n_z^2&0\\
            0&n_x^2+n_y^2+n_z^2
        \end{matrix}\right]=\left[\begin{matrix}
            1&0\\
            0&1
        \end{matrix}\right]=I.
    \end{gather}

    Using the above result,
    \begin{align}
        \notag R_{\hat{n}}(\theta)=&\exp(-i\theta\hat{n}\cdot\vec{\sigma}/2)=\sum_{k=0}^{\infty}\frac{1}{k!}(-i\theta\hat{n}\cdot\vec{\sigma}/2)=\sum_{k=0}^{\infty}\frac{1}{(2k)!}(-i\theta\hat{n}\cdot\vec{\sigma}/2)^{2k}+\sum_{k=0}^{\infty}\frac{1}{(2k+1)!}(-i\theta\hat{n}\cdot\vec{\sigma}/2)^{2k+1}\\
        \notag=&\sum_{k=0}^{\infty}\frac{(-1)^k}{(2k)!}\left(\frac{\theta}{2}\right)^{2k}+\sum_{k=0}^{\infty}\frac{-i(-1)^k}{(2k+1)!}\left(\frac{\theta}{2}\right)^{2k+1}(\hat{n}\cdot\vec{\sigma})=\cos\left(\frac{\theta}{2}\right)I-i\sin\left(\frac{\theta}{2}\right)(\hat{n}\cdot\vec{\sigma})\\
        =&\cos\left(\frac{\theta}{2}\right)I-i\sin\left(\frac{\theta}{2}\right)(n_xX+n_yY+n_zZ),
    \end{align}
    which verifies Equation (4.8).
\end{pf}

\begin{exe}[Bloch sphere interpretation of rotations]
    One reason why the $R_{\hat{n}}(\theta)$ operators are referred to as rotation operators is the following fact, which you are to prove. Suppose a single qubit has a state represented by Bloch vector $\vec{\lambda}$. Then the effect of the rotation $R_{\hat{n}}(\theta)$ on the state is to rotate it by an angle $\theta$ about the $\hat{n}$ axis of the Bloch sphere. This fact explains the rather mysterious looking factor of two in the definition of the rotation matrices.
\end{exe}
\begin{pf}
    \footnote{\url{http://www.vcpc.univie.ac.at/~ian/hotlist/qc/talks/bloch-sphere-rotations.pdf}}Suppose $\hat{n}$ forms angle $\theta$ with the $z$ axis and the projection of $\hat{n}$ onto the $xy$ plane forms angle $\varphi$ with the $x$ axis, i.e.
    \begin{align}
        \label{E4.8}
        \hat{n}=n_x\hat{x}+n_y\hat{y}+n_z\hat{z}=\sin\theta\cos\varphi\hat{x}+\sin\theta\sin\varphi\hat{y}+\cos\theta\hat{z}.
    \end{align}
    Rotating the Bloch vector $\vec{\lambda}$ about $\hat{n}$ axis by angle $\alpha$ is equivalent to the following operation sequence:
    \begin{itemize}
        \item[1.] Rotating about $z$ axis by angle $-\varphi$ so that $\hat{n}$ is rotated onto $xz$ plane;
        \item[2.] Rotating about $y$ axis by angle $-\theta$ so that $\hat{n}$ coincides with $z$ axis;
        \item[3.] Rotating about $z$ axis by angle $\alpha$ so $\vec{\lambda}$ is rotated about $\hat{n}$ axis by angle $\alpha$;
        \item[4.] Rotating about $y$ axis by angle $\theta$ to counteract operation step 2.;
        \item[5.] Rotating about $z$ axis by angle $\theta$ to counteract operation step 1. so that $\hat{n}$ is rotated back to its original position,
    \end{itemize}
    i.e.,
    \begin{align}
        \notag&R_z(\varphi)R_y(\theta)R_z(\alpha)R_y(-\theta)R_z(-\varphi)\\
        \notag=&\left[\cos\frac{\varphi}{2}I-i\sin\frac{\varphi}{2}Z\right]\left[\cos\frac{\theta}{2}I-i\sin\frac{\theta}{2}Y\right]\left[\cos\frac{\alpha}{2}I-i\sin\frac{\alpha}{2}Z\right]\left[\cos\frac{\theta}{2}I+i\sin\frac{\theta}{2}Y\right]\left[\cos\frac{\varphi}{2}I+i\sin\frac{\varphi}{2}Z\right]\\
        \notag=&\left[\cos\frac{\varphi}{2}I-i\sin\frac{\varphi}{2}Z\right]\left[\cos\frac{\theta}{2}\cos\frac{\alpha}{2}I-i\cos\frac{\theta}{2}\sin\frac{\alpha}{2}Z-i\sin\frac{\theta}{2}\cos\frac{\alpha}{2}Y-\sin\frac{\theta}{2}\sin\frac{\alpha}{2}YZ\right]\left[\cos\frac{\theta}{2}I+i\sin\frac{\theta}{2}Y\right]\\
        \notag&\left[\cos\frac{\varphi}{2}I+i\sin\frac{\varphi}{2}Z\right]\\
        \notag=&\left[\cos\frac{\varphi}{2}I-i\sin\frac{\varphi}{2}Z\right]\left[\cos\frac{\theta}{2}\cos\frac{\alpha}{2}I-i\cos\frac{\theta}{2}\sin\frac{\alpha}{2}Z-i\sin\frac{\theta}{2}\cos\frac{\alpha}{2}Y-i\sin\frac{\theta}{2}\sin\frac{\alpha}{2}X\right]\left[\cos\frac{\theta}{2}I+i\sin\frac{\theta}{2}Y\right]\\
        \notag&\left[\cos\frac{\varphi}{2}I+i\sin\frac{\varphi}{2}Z\right]\\
        \notag=&\left[\cos\frac{\varphi}{2}I-i\sin\frac{\varphi}{2}Z\right]\left[\cos^2\frac{\theta}{2}\cos\frac{\alpha}{2}I-i\cos^2\frac{\theta}{2}\sin\frac{\alpha}{2}Z-i\sin\frac{\theta}{2}\cos\frac{\theta}{2}\cos\frac{\alpha}{2}Y-i\sin\frac{\theta}{2}\cos\frac{\theta}{2}\sin\frac{\alpha}{2}X\right.\\
        &\left.+i\cos\frac{\theta}{2}\sin\frac{\theta}{2}\cos\frac{\alpha}{2}Y+\cos\frac{\theta}{2}\sin\frac{\theta}{2}\sin\frac{\alpha}{2}ZY+\sin^2\frac{\theta}{2}\cos\frac{\alpha}{2}Y^2+\sin^2\frac{\theta}{2}\sin\frac{\alpha}{2}XY\right]\left[\cos\frac{\varphi}{2}I+i\sin\frac{\varphi}{2}Z\right]\\
        \notag=&\left[\cos\frac{\varphi}{2}I-i\sin\frac{\varphi}{2}Z\right]\left[\cos^2\frac{\theta}{2}\cos\frac{\alpha}{2}I-i\cos^2\frac{\theta}{2}\sin\frac{\alpha}{2}Z-i\sin\frac{\theta}{2}\cos\frac{\theta}{2}\cos\frac{\alpha}{2}Y-i\sin\frac{\theta}{2}\cos\frac{\theta}{2}\sin\frac{\alpha}{2}X\right.\\
        \notag&\left.+i\cos\frac{\theta}{2}\sin\frac{\theta}{2}\cos\frac{\alpha}{2}Y-i\cos\frac{\theta}{2}\sin\frac{\theta}{2}\sin\frac{\alpha}{2}X+\sin^2\frac{\theta}{2}\cos\frac{\alpha}{2}I+i\sin^2\frac{\theta}{2}\sin\frac{\alpha}{2}Z\right]\left[\cos\frac{\varphi}{2}I+i\sin\frac{\varphi}{2}Z\right]\\
        \notag=&\left[\cos\frac{\varphi}{2}I-i\sin\frac{\varphi}{2}Z\right]\left[\cos\frac{\alpha}{2}I-i\cos\theta\sin\frac{\alpha}{2}Z-i\sin\theta\sin\frac{\alpha}{2}X\right]\left[\cos\frac{\varphi}{2}I+i\sin\frac{\varphi}{2}Z\right]\\
        \notag=&\left[\cos\frac{\varphi}{2}\cos\frac{\alpha}{2}I-i\cos\frac{\varphi}{2}\cos\theta\sin\frac{\alpha}{2}Z-i\cos\frac{\varphi}{2}\sin\theta\sin\frac{\alpha}{2}X-i\sin\frac{\varphi}{2}\cos\frac{\alpha}{2}Z-\sin\frac{\varphi}{2}\cos\theta\sin\frac{\alpha}{2}Z^2\right.\\
        \notag&\left.-\sin\frac{\varphi}{2}\sin\theta\sin\frac{\alpha}{2}ZX\right]\left[\cos\frac{\varphi}{2}I+i\sin\frac{\varphi}{2}Z\right]\\
        \notag=&\left[\cos\frac{\varphi}{2}\cos\frac{\alpha}{2}I-i\cos\frac{\varphi}{2}\cos\theta\sin\frac{\alpha}{2}Z-i\cos\frac{\varphi}{2}\sin\theta\sin\frac{\alpha}{2}X-i\sin\frac{\varphi}{2}\cos\frac{\alpha}{2}Z-\sin\frac{\varphi}{2}\cos\theta\sin\frac{\alpha}{2}I-i\sin\frac{\varphi}{2}\sin\theta\sin\frac{\alpha}{2}Y\right]\\
        \notag&\left[\cos\frac{\varphi}{2}I+i\sin\frac{\varphi}{2}Z\right]\\
        \notag=&\left[\left(\cos\frac{\varphi}{2}\cos\frac{\alpha}{2}-\sin\frac{\varphi}{2}\cos\theta\sin\frac{\alpha}{2}\right)I-i\left(\cos\frac{\varphi}{2}\cos\theta\sin\frac{\alpha}{2}+\sin\frac{\varphi}{2}\cos\frac{\alpha}{2}\right)Z-i\cos\frac{\varphi}{2}\sin\theta\sin\frac{\alpha}{2}X\right.\\
        \notag&\left.-i\sin\frac{\varphi}{2}\sin\theta\sin\frac{\alpha}{2}Y\right]\left[\cos\frac{\varphi}{2}I+i\sin\frac{\varphi}{2}Z\right]\\
        \notag=&\cos\frac{\varphi}{2}\left(\cos\frac{\varphi}{2}\cos\frac{\alpha}{2}-\sin\frac{\varphi}{2}\cos\theta\sin\frac{\alpha}{2}\right)I-i\cos\frac{\varphi}{2}\left(\cos\frac{\varphi}{2}\cos\theta\sin\frac{\alpha}{2}+\sin\frac{\varphi}{2}\cos\frac{\alpha}{2}\right)Z-i\cos^2\frac{\varphi}{2}\sin\theta\sin\frac{\alpha}{2}X\\
        \notag&-i\sin\frac{\varphi}{2}\cos\frac{\varphi}{2}\sin\theta\sin\frac{\alpha}{2}Y+i\sin\frac{\varphi}{2}\left(\cos\frac{\varphi}{2}\cos\frac{\alpha}{2}-\sin\frac{\varphi}{2}\cos\theta\sin\frac{\alpha}{2}\right)Z+\sin\frac{\varphi}{2}\left(\cos\frac{\varphi}{2}\cos\theta\sin\frac{\alpha}{2}+\sin\frac{\varphi}{2}\cos\frac{\alpha}{2}\right)Z^2\\
        \notag&+\cos\frac{\varphi}{2}\sin\frac{\varphi}{2}\sin\theta\sin\frac{\alpha}{2}XZ+\sin^2\frac{\varphi}{2}\sin\theta\sin\frac{\alpha}{2}YZ\\
        \notag=&\cos\frac{\varphi}{2}\left(\cos\frac{\varphi}{2}\cos\frac{\alpha}{2}-\sin\frac{\varphi}{2}\cos\theta\sin\frac{\alpha}{2}\right)I-i\cos\frac{\varphi}{2}\left(\cos\frac{\varphi}{2}\cos\theta\sin\frac{\alpha}{2}+\sin\frac{\varphi}{2}\cos\frac{\alpha}{2}\right)Z-i\cos^2\frac{\varphi}{2}\sin\theta\sin\frac{\alpha}{2}X\\
        \notag&-i\sin\frac{\varphi}{2}\cos\frac{\varphi}{2}\sin\theta\sin\frac{\alpha}{2}Y+i\sin\frac{\varphi}{2}\left(\cos\frac{\varphi}{2}\cos\frac{\alpha}{2}-\sin\frac{\varphi}{2}\cos\theta\sin\frac{\alpha}{2}\right)Z+\sin\frac{\varphi}{2}\left(\cos\frac{\varphi}{2}\cos\theta\sin\frac{\alpha}{2}+\sin\frac{\varphi}{2}\cos\frac{\alpha}{2}\right)I\\
        \notag&-i\cos\frac{\varphi}{2}\sin\frac{\varphi}{2}\sin\theta\sin\frac{\alpha}{2}Y+i\sin^2\frac{\varphi}{2}\sin\theta\sin\frac{\alpha}{2}X\\
        =&\cos\frac{\alpha}{2}I-i\sin\frac{\alpha}{2}\left(\cos\varphi\sin\theta X+\sin\varphi\sin\theta Y+\cos\theta Z\right)=\cos\frac{\alpha}{2}I-i(n_xX+n_yY+n_zZ),
    \end{align}
    which equals $R_{\hat{n}}(\alpha)$.
    Therefore, the effect of the rotation $R_{\hat{n}}(\theta)$ on the state is to rotate it by an angle $\theta$ about the $\hat{n}$ axis of the Bloch sphere.
\end{pf}

\begin{exe}
    Show that $XYX=-Y$ and use this to prove that $XR_y(\theta)X=R_y(-\theta)$.
\end{exe}
\begin{pf}
    \begin{align}
        XYX=iZX=i^2Y=-Y.
    \end{align}
    In this way,
    \begin{align}
        \notag XR_y(\theta)X=&X\left(\cos\frac{\theta}{2}I-i\sin\frac{\theta}{2}Y\right)X=\left(\cos\frac{\theta}{2}X-i\sin\frac{\theta}{2}XY\right)X=\left(\cos\frac{\theta}{2}X+\sin\frac{\theta}{2}Z\right)X=\cos\frac{\theta}{2}X^2+\sin\frac{\theta}{2}ZX\\
        =&\cos\frac{\theta}{2}I+i\sin\frac{\theta}{2}Y=R_y(-\theta).
    \end{align}
\end{pf}

\begin{exe}
    An arbitrary single qubit unitary operator can be written in the form
    \begin{align}
        U=\exp(i\alpha)R_{\hat{n}}(\theta)
    \end{align}
    for some real number $\alpha$ and $\theta$, and a real three-dimensional unit vector $\hat{n}$.
    \begin{itemize}
        \item[1.] Prove this fact.
        \item[2.] Find values for $\alpha$, $\theta$, and $\hat{n}$ giving the Hadamard gate $H$.
        \item[3.] Find value for $\alpha$, $\theta$, and $\hat{n}$ giving the phase gate
        \begin{align}
            S=\left[\begin{matrix}
                1&0\\
                0&i
            \end{matrix}\right].
        \end{align}
    \end{itemize}
\end{exe}
\begin{sol}
    \begin{itemize}
        \item[1.] Suppose an arbitrary qubit unitary operator is
        \begin{align}
            U=\left[\begin{matrix}
                a&b\\
                c&d
            \end{matrix}\right].
        \end{align}
        where $a$, $b$, $c$, and $d$ are complex numbers. Since $U$ is unitary,
        \begin{gather}
            UU^{\dagger}=\left[\begin{matrix}
                a&b\\
                c&d
            \end{matrix}\right]\left[\begin{matrix}
                a^*&c^*\\
                b^*&d^*
            \end{matrix}\right]=\left[\begin{matrix}
                aa^*+bb^*&ac^*+bd^*\\
                a^*c+b^*d&cc^*+dd^*
            \end{matrix}\right]=I=\left[\begin{matrix}
                1&0\\
                0&1
            \end{matrix}\right],\\
            \Longrightarrow aa^*+bb^*=cc^*+dd^*=1,\quad ac^*+bd^*=0,\\
            \Longrightarrow c=\mp b^*,\quad d=\pm a^*,\quad\abs{a}^2+\abs{b}^2=1.
        \end{gather}
        Hence $U$ can be written as
        \begin{align}
            U=\left[\begin{matrix}
                a&b\\
                \mp b^*&\pm a^*
            \end{matrix}\right]=\left\{\begin{array}{ll}
                \left[\begin{matrix}
                    a&b\\
                    -b^*&a^*
                \end{matrix}\right],&\text{for }\det(U)=1;\\
                \left[\begin{matrix}
                    a&b\\
                    b^*&-a^*
                \end{matrix}\right],&\text{for }\det(U)=-1,\\
            \end{array}\right.
        \end{align}
        where $\abs{a}^2+\abs{b}^2=1$.
        Without loss of generality, we can find real number $\alpha$, and $\theta$, and a real three-dimensional unit vector $\hat{n}=n_x\hat{x}+n_y\hat{y}+z\hat{z}$ so that
        \begin{align}
            a=&e^{i\alpha}\left(\cos\frac{\theta}{2}-in_z\sin\frac{\theta}{2}\right),\\
            b=&-e^{i\alpha}(in_x+n_y)\sin\frac{\theta}{2},\\
            e^{i\alpha}=&\left\{\begin{array}{ll}
                1,&\text{for }\det(U)=1;\\
                i,&\text{for }\det(U)=-1,\\
            \end{array}\right.
        \end{align}
        which satisfies
        \begin{align}
            \abs{a}^2+\abs{b}^2=\cos^2\frac{\theta}{2}+n_z^2\sin^2\frac{\theta}{2}+n_x^2\sin^2\frac{\theta}{2}+n_y^2\sin^2\frac{\theta}{2}=\cos^2\frac{\theta}{2}+\sin^2\frac{\theta}{2}=1,
        \end{align}
        for $n_x^2+n_y^2+n_z^2=1$.
        In this way, the single qubit unitary operator can be written in the form
        \begin{align}
            \notag U=&\left[\begin{matrix}
                a&b\\
                \mp b^*&\pm a^*
            \end{matrix}\right]=\left[\begin{matrix}
                e^{i\alpha}\left(\cos\frac{\theta}{2}-in_z\sin\frac{\theta}{2}\right)&-e^{i\alpha}(in_x+n_y)\sin\frac{\theta}{2}\\
                \pm e^{-i\alpha}(-in_x+n_y)\sin\frac{\theta}{2}&\pm e^{-i\alpha}\left(\cos\frac{\theta}{2}+in_z\sin\frac{\theta}{2}\right)
            \end{matrix}\right]\\
            \notag=&\left\{\begin{array}{ll}
                \left[\begin{matrix}
                    e^{i\alpha}\left(\cos\frac{\theta}{2}-in_z\sin\frac{\theta}{2}\right)&-e^{i\alpha}(in_x+n_y)\sin\frac{\theta}{2}\\
                    e^{-i\alpha}(-in_x+n_y)\sin\frac{\theta}{2}&e^{-i\alpha}\left(\cos\frac{\theta}{2}+in_z\sin\frac{\theta}{2}\right)
                \end{matrix}\right],&\text{for }\det(U)=1\\
                \left[\begin{matrix}
                    e^{i\alpha}\left(\cos\frac{\theta}{2}-in_z\sin\frac{\theta}{2}\right)&-e^{i\alpha}(in_x+n_y)\sin\frac{\theta}{2}\\
                    -e^{-i\alpha}(-in_x+n_y)\sin\frac{\theta}{2}&-e^{-i\alpha}\left(\cos\frac{\theta}{2}+in_z\sin\frac{\theta}{2}\right)
                \end{matrix}\right],&\text{for }\det(U)=-1
            \end{array}\right.\\
            \notag=&\left\{\begin{array}{ll}
                \left[\begin{matrix}
                    \cos\frac{\theta}{2}-in_z\sin\frac{\theta}{2}&-(in_x+n_y)\sin\frac{\theta}{2}\\
                    (-in_x+n_y)\sin\frac{\theta}{2}&\cos\frac{\theta}{2}+in_z\sin\frac{\theta}{2}
                \end{matrix}\right],&\text{for }\det(U)=1\\
                \left[\begin{matrix}
                    i\left(\cos\frac{\theta}{2}-in_z\sin\frac{\theta}{2}\right)&-i(in_x+n_y)\sin\frac{\theta}{2}\\
                    i(-in_x+n_y)\sin\frac{\theta}{2}&i\left(\cos\frac{\theta}{2}+in_z\sin\frac{\theta}{2}\right)
                \end{matrix}\right],&\text{for }\det(U)=-1
            \end{array}\right.\\
            \notag=&\left\{\begin{array}{ll}
                \left[\begin{matrix}
                    \cos\frac{\theta}{2}-in_z\sin\frac{\theta}{2}&-(in_x+n_y)\sin\frac{\theta}{2}\\
                    (-in_x+n_y)\sin\frac{\theta}{2}&\cos\frac{\theta}{2}+in_z\sin\frac{\theta}{2}
                \end{matrix}\right],&\text{for }\det(U)=1\\
                i\left[\begin{matrix}
                    \cos\frac{\theta}{2}-in_z\sin\frac{\theta}{2}&-(in_x+n_y)\sin\frac{\theta}{2}\\
                    (-in_x+n_y)\sin\frac{\theta}{2}&\cos\frac{\theta}{2}+in_z\sin\frac{\theta}{2}
                \end{matrix}\right],&\text{for }\det(U)=-1
            \end{array}\right.\\
            \notag=&e^{i\alpha}\left[\begin{matrix}
                \cos\frac{\theta}{2}-in_z\sin\frac{\theta}{2}&-(in_x+n_y)\sin\frac{\theta}{2}\\
                (-in_x+n_y)\sin\frac{\theta}{2}&\cos\frac{\theta}{2}+in_z\sin\frac{\theta}{2}
            \end{matrix}\right]\\
            \notag=&e^{i\alpha}\left\{\cos\theta\left[\begin{matrix}
                1&0\\
                0&1
            \end{matrix}\right]-i\sin\theta\left(n_x\left[\begin{matrix}
                0&1\\
                1&0
            \end{matrix}\right]+n_y\left[\begin{matrix}
                0&-i\\
                i&0
            \end{matrix}\right]+n_z\left[\begin{matrix}
                1&0\\
                0&-1
            \end{matrix}\right]\right)\right\}\\
            \notag=&e^{i\alpha}\left[\cos\frac{\theta}{2}I-i\sin\frac{\theta}{2}\left(n_xX+n_yY+n_zZ\right)\right]\\
            =&\exp(i\alpha)R_{\hat{n}}(\theta).
        \end{align}
        \item[2.] The Hadamard gate is
        \begin{align}
            H=\frac{1}{\sqrt{2}}\left[\begin{matrix}
                1&1\\
                1&-1
            \end{matrix}\right],
        \end{align}
        whose determinant is
        \begin{align}
            \det(H)=\abs{\begin{matrix}
                \frac{1}{\sqrt{2}}&\frac{1}{\sqrt{2}}\\
                \frac{1}{\sqrt{2}}&-\frac{1}{\sqrt{2}}
            \end{matrix}}=-1,
        \end{align}
        so we choose
        \begin{gather}
            \begin{align}
                a=&e^{i\alpha}\left(\cos\frac{\theta}{2}-in_z\sin\frac{\theta}{2}\right)=\frac{1}{\sqrt{2}},\\
                b=&-e^{i\alpha}(in_x+n_y)\sin\frac{\theta}{2}=\frac{1}{\sqrt{2}},\\
                e^{i\alpha}=&i,
            \end{align}\\
            \Longrightarrow\alpha=\frac{\pi}{2},\quad\theta=\pi,\quad\hat{n}=n_x\hat{x}+n_y\hat{y}+n_z\hat{z}=\frac{1}{\sqrt{2}}\hat{x}+\frac{1}{\sqrt{2}}\hat{z}.
        \end{gather}
        \item[3.] For phase gate $S$,
        \begin{gather}
            \begin{align}
                e^{i\alpha}\left(\cos\frac{\theta}{2}-in_z\sin\frac{\theta}{2}\right)=&1,\\
                -e^{i\alpha}\left(in_x+n_y\right)\sin\frac{\theta}{2}=0,\\
                e^{-i\alpha}(-in_x+n_y)\sin\frac{\theta}{2}=0,\\
                e^{i\alpha}\left(\cos\frac{\theta}{2}+in_z\sin\frac{\theta}{2}\right)=i,
            \end{align}\\
            \Longrightarrow\alpha=\frac{\pi}{4},\quad\theta=\frac{\pi}{2},\quad\hat{n}=\hat{z}.
        \end{gather}
    \end{itemize}
\end{sol}

\begin{exe}
    Explain why any single qubit unitary operator may be written in the form (4.12)\footnote{$U=\left[\begin{matrix}
        e^{i(\alpha-\beta/2-\delta/2)}\cos\frac{\gamma}{2}&-e^{i(\alpha-\beta/2+\delta/2)}\sin\frac{\gamma}{2}\\
        e^{i(\alpha+\beta/2-\delta/2)}\sin\frac{\gamma}{2}&e^{i(\alpha+\beta/2+\delta/2)}\cos\frac{\gamma}{2}
    \end{matrix}\right]$.}.
\end{exe}
\begin{sol}
    Suppose an arbitrary qubit unitary operator is
        \begin{align}
            U=\left[\begin{matrix}
                a&b\\
                c&d
            \end{matrix}\right].
        \end{align}
        where $a$, $b$, $c$, and $d$ are complex numbers. Since $U$ is unitary,
        \begin{gather}
            UU^{\dagger}=\left[\begin{matrix}
                a&b\\
                c&d
            \end{matrix}\right]\left[\begin{matrix}
                a^*&c^*\\
                b^*&d^*
            \end{matrix}\right]=\left[\begin{matrix}
                aa^*+bb^*&ac^*+bd^*\\
                a^*c+b^*d&cc^*+dd^*
            \end{matrix}\right]=I=\left[\begin{matrix}
                1&0\\
                0&1
            \end{matrix}\right],\\
            \Longrightarrow aa^*+bb^*=cc^*+dd^*=1,\quad ac^*+bd^*=0.
        \end{gather}
        Since
        \begin{align}
            \notag&\left[\begin{matrix}
                e^{i(\alpha-\beta/2-\delta/2)}\cos\frac{\gamma}{2}&-e^{i(\alpha-\beta/2+\delta/2)}\sin\frac{\gamma}{2}\\
                e^{i(\alpha+\beta/2-\delta/2)}\sin\frac{\gamma}{2}&e^{i(\alpha+\beta/2+\delta/2)}\cos\frac{\gamma}{2}
            \end{matrix}\right]\left[\begin{matrix}
                e^{i(\alpha-\beta/2-\delta/2)}\cos\frac{\gamma}{2}&-e^{i(\alpha-\beta/2+\delta/2)}\sin\frac{\gamma}{2}\\
                e^{i(\alpha+\beta/2-\delta/2)}\sin\frac{\gamma}{2}&e^{i(\alpha+\beta/2+\delta/2)}\cos\frac{\gamma}{2}
            \end{matrix}\right]^{\dagger}\\
            \notag=&\left[\begin{matrix}
                e^{i(\alpha-\beta/2-\delta/2)}\cos\frac{\gamma}{2}&-e^{i(\alpha-\beta/2+\delta/2)}\sin\frac{\gamma}{2}\\
                e^{i(\alpha+\beta/2-\delta/2)}\sin\frac{\gamma}{2}&e^{i(\alpha+\beta/2+\delta/2)}\cos\frac{\gamma}{2}
            \end{matrix}\right]\left[\begin{matrix}
                e^{-i(\alpha-\beta/2-\delta/2)}\cos\frac{\gamma}{2}&e^{-i(\alpha+\beta/2-\delta/2)}\sin\frac{\gamma}{2}\\
                -e^{-i(\alpha-\beta/2+\delta/2)}\sin\frac{\gamma}{2}&e^{-i(\alpha+\beta/2+\delta/2)}\cos\frac{\gamma}{2}
            \end{matrix}\right]\\
            \notag=&\left[\begin{matrix}
                e^{i(\alpha-\beta/2-\delta/2)}\cos\frac{\gamma}{2}e^{-i(\alpha-\beta/2-\delta/2)}\cos\frac{\gamma}{2}+e^{i(\alpha-\beta/2+\delta/2)}\sin\frac{\gamma}{2}e^{-i(\alpha-\beta/2+\delta/2)}\sin\frac{\gamma}{2},\\
                e^{i(\alpha-\beta/2-\delta/2)}\cos\frac{\gamma}{2}e^{-i(\alpha+\beta/2-\delta/2)}\sin\frac{\gamma}{2}-e^{i(\alpha-\beta/2+\delta/2)}\sin\frac{\gamma}{2}e^{-i(\alpha+\beta/2+\delta/2)}\cos\frac{\gamma}{2};\\
                e^{i(\alpha+\beta/2-\delta/2)}\sin\frac{\gamma}{2}e^{-i(\alpha-\beta/2-\delta/2)}\cos\frac{\gamma}{2}-e^{i(\alpha+\beta/2+\delta/2)}\cos\frac{\gamma}{2}e^{-i(\alpha-\beta/2+\delta/2)}\sin\frac{\gamma}{2},\\
                e^{i(\alpha+\beta/2-\delta/2)}\sin\frac{\gamma}{2}e^{-i(\alpha+\beta/2-\delta/2)}\sin\frac{\gamma}{2}+e^{i(\alpha+\beta/2+\delta/2)}\cos\frac{\gamma}{2}e^{-i(\alpha+\beta/2+\delta/2)}\cos\frac{\gamma}{2}
            \end{matrix}\right]\\
            =&\left[\begin{matrix}
                1&0\\
                0&1
            \end{matrix}\right],
        \end{align}
        the matrix $\left[\begin{matrix}
            e^{i(\alpha-\beta/2-\delta/2)}\cos\frac{\gamma}{2}&-e^{i(\alpha-\beta/2+\delta/2)}\sin\frac{\gamma}{2}\\
            e^{i(\alpha+\beta/2-\delta/2)}\sin\frac{\gamma}{2}&e^{i(\alpha+\beta/2+\delta/2)}\cos\frac{\gamma}{2}
        \end{matrix}\right]$ is also unitary.
        For arbitrary $a$, $b$, $c$, and $d$ such that $U$ is unitary, let
        \begin{align}
            a=&e^{i(\alpha-\beta/2-\delta/2)}\cos\frac{\gamma}{2},\\
            b=&-e^{i(\alpha-\beta/2+\delta/2)}\sin\frac{\gamma}{2},\\
            c=&e^{i(\alpha+\beta/2-\delta/2)}\sin\frac{\gamma}{2},\\
            d=&e^{i(\alpha+\beta/2+\delta/2)}\cos\frac{\gamma}{2},
        \end{align}
        i.e., set
        \begin{align}
            \alpha=&-\frac{i}{2}\ln(ad-bc),\\
            \beta=&\frac{i}{2}\ln\left(-\frac{ab}{cd}\right),\\
            \delta=&\frac{i}{2}\ln\left(-\frac{ac}{bd}\right),\\
            \gamma=&\arccos\abs{ad+bc}.
        \end{align}
        Then
        \begin{align}
            U=\left[\begin{matrix}
                e^{i(\alpha-\beta/2-\delta/2)}\cos\frac{\gamma}{2}&-e^{i(\alpha-\beta/2+\delta/2)}\sin\frac{\gamma}{2}\\
                e^{i(\alpha+\beta/2-\delta/2)}\sin\frac{\gamma}{2}&e^{i(\alpha+\beta/2+\delta/2)}\cos\frac{\gamma}{2}
            \end{matrix}\right].
        \end{align}
        Therefore, any single unitary operator may be written in the form (4.12).
\end{sol}

\begin{exe}[X-Y decomposition of rotations]
    Give a decomposition analogous to Theorem 4.1\footnote{($Z-Y$ decomposition for single qubit) Suppose $U$ is a unitary operation on a single qubit. Then there exist real numbers $\alpha$, $\beta$, $\gamma$ and $\delta$ such that $U=e^{i\alpha}R_z(\beta)R_y(\gamma)R_z(\delta)$.} but using $R_x$ instead of $R_z$.
\end{exe}
\begin{sol}
    Suppose an arbitrary qubit unitary operator is
    \begin{align}
        U=\left[\begin{matrix}
            a&b\\
            c&d
        \end{matrix}\right],
    \end{align}
    where $a$, $b$, $c$, and $d$ are complex numbers. Since $U$ is unitary,
    \begin{gather}
        UU^{\dagger}=\left[\begin{matrix}
            a&b\\
            c&d
        \end{matrix}\right]\left[\begin{matrix}
            a^*&c^*\\
            b^*&d^*
        \end{matrix}\right]=\left[\begin{matrix}
            aa^*+bb^*&ac^*+bd^*\\
            a^*c+b^*d&cc^*+dd^*
        \end{matrix}\right]=I=\left[\begin{matrix}
            1&0\\
            0&1
        \end{matrix}\right],\\
        \Longrightarrow aa^*+bb^*=cc^*+dd^*=1,\quad ac^*+bd^*=0.
    \end{gather}
    The $X-Y$ decomposition of rotation is
    \begin{align}
        \notag&e^{i\alpha}R_z(\beta)R_x(\gamma)R_z(\delta)=e^{i\alpha}\left[\begin{matrix}
            \cos\frac{\beta}{2}&-i\sin\frac{\beta}{2}\\
            -i\sin\frac{\beta}{2}&\cos\frac{\beta}{2}
        \end{matrix}\right]\left[\begin{matrix}
            \cos\frac{\gamma}{2}&-\sin\frac{\gamma}{2}\\
            \sin\frac{\gamma}{2}&\cos\frac{\gamma}{2}
        \end{matrix}\right]\left[\begin{matrix}
            \cos\frac{\delta}{2}&-i\sin\frac{\delta}{2}\\
            -i\sin\frac{\delta}{2}&\cos\frac{\delta}{2}
        \end{matrix}\right]\\
        \notag&=e^{i\alpha}\left[\begin{matrix}
            \cos\frac{\beta}{2}\cos\frac{\gamma}{2}-i\sin\frac{\beta}{2}\sin\frac{\gamma}{2}&-\cos\frac{\beta}{2}\sin\frac{\gamma}{2}-i\sin\frac{\beta}{2}\cos\frac{\gamma}{2}\\
            -i\sin\frac{\beta}{2}\cos\frac{\gamma}{2}+\cos\frac{\beta}{2}\sin\frac{\gamma}{2}&i\sin\frac{\beta}{2}\sin\frac{\gamma}{2}+\cos\frac{\beta}{2}\cos\frac{\gamma}{2}
        \end{matrix}\right]\left[\begin{matrix}
            \cos\frac{\delta}{2}&-i\sin\frac{\delta}{2}\\
            -i\sin\frac{\delta}{2}&\cos\frac{\delta}{2}
        \end{matrix}\right]\\
        \notag&=e^{i\alpha}\\
        \notag&\left[\begin{smallmatrix}
            \left(\cos\frac{\beta}{2}\cos\frac{\gamma}{2}-i\sin\frac{\beta}{2}\sin\frac{\gamma}{2}\right)\cos\frac{\delta}{2}-i\left(-\cos\frac{\beta}{2}\sin\frac{\gamma}{2}-i\sin\frac{\beta}{2}\cos\frac{\gamma}{2}\right)\sin\frac{\delta}{2}&-i\left(\cos\frac{\beta}{2}\cos\frac{\gamma}{2}-i\sin\frac{\beta}{2}\sin\frac{\gamma}{2}\right)\sin\frac{\delta}{2}+\left(-\cos\frac{\beta}{2}\sin\frac{\gamma}{2}-i\sin\frac{\beta}{2}\cos\frac{\gamma}{2}\right)\cos\frac{\delta}{2}\\
            \left(-i\sin\frac{\beta}{2}\cos\frac{\gamma}{2}+\cos\frac{\beta}{2}\sin\frac{\gamma}{2}\right)\cos\frac{\delta}{2}-i\left(i\sin\frac{\beta}{2}\sin\frac{\gamma}{2}+\cos\frac{\beta}{2}\cos\frac{\gamma}{2}\right)\sin\frac{\delta}{2}&-i\left(-i\sin\frac{\beta}{2}\cos\frac{\gamma}{2}+\cos\frac{\beta}{2}\sin\frac{\gamma}{2}\right)\sin\frac{\delta}{2}+\left(i\sin\frac{\beta}{2}\sin\frac{\gamma}{2}+\cos\frac{\beta}{2}\cos\frac{\gamma}{2}\right)\cos\frac{\delta}{2}
        \end{smallmatrix}\right]\\
        \notag&=e^{i\alpha}\\
        \notag&\left[\begin{smallmatrix}
            \cos\frac{\gamma}{2}\left(\cos\frac{\beta}{2}\cos\frac{\delta}{2}-\sin\frac{\beta}{2}\sin\frac{\delta}{2}\right)+i\sin\frac{\gamma}{2}\left(-\sin\frac{\beta}{2}\cos\frac{\delta}{2}+\cos\frac{\beta}{2}\sin\frac{\delta}{2}\right)&\sin\frac{\gamma}{2}\left(-\sin\frac{\beta}{2}\sin\frac{\delta}{2}-\cos\frac{\beta}{2}\cos\frac{\delta}{2}\right)+i\cos\frac{\gamma}{2}\left(-\cos\frac{\beta}{2}\sin\frac{\delta}{2}-\sin\frac{\beta}{2}\cos\frac{\delta}{2}\right)\\
            \sin\frac{\gamma}{2}\left(\cos\frac{\beta}{2}\cos\frac{\delta}{2}+\sin\frac{\beta}{2}\sin\frac{\delta}{2}\right)+i\cos\frac{\gamma}{2}\left(-\sin\frac{\beta}{2}\cos\frac{\delta}{2}-\cos\frac{\beta}{2}\sin\frac{\delta}{2}\right)&\cos\frac{\gamma}{2}\left(-\sin\frac{\beta}{2}\sin\frac{\delta}{2}+\cos\frac{\beta}{2}\cos\frac{\delta}{2}\right)+i\sin\frac{\gamma}{2}\left(-\cos\frac{\beta}{2}\sin\frac{\delta}{2}+\sin\frac{\beta}{2}\cos\frac{\delta}{2}\right)
        \end{smallmatrix}\right]\\
        \notag&=e^{i\alpha}\left[\begin{matrix}
            \cos\frac{\gamma}{2}\cos\frac{\beta+\delta}{2}-i\sin\frac{\gamma}{2}\sin\frac{\beta-\delta}{2}&-\sin\frac{\gamma}{2}\cos\frac{\beta-\delta}{2}-i\cos\frac{\gamma}{2}\sin\frac{\beta+\delta}{2}\\
            \sin\frac{\gamma}{2}\cos\frac{\beta-\delta}{2}-i\cos\frac{\gamma}{2}\sin\frac{\beta+\delta}{2}&\cos\frac{\gamma}{2}\cos\frac{\beta+\delta}{2}+i\sin\frac{\gamma}{2}\sin\frac{\beta-\delta}{2}
        \end{matrix}\right]\\
    \end{align}
    Since
    \begin{align}
        \notag&e^{i\alpha}R_z(\beta)R_x(\gamma)R_z(\delta)[e^{i\alpha}R_z(\beta)R_x(\gamma)R_z(\delta)]^{\dagger}\\
        \notag=&e^{i\alpha}\left[\begin{matrix}
            \cos\frac{\gamma}{2}\cos\frac{\beta+\delta}{2}-i\sin\frac{\gamma}{2}\sin\frac{\beta-\delta}{2}&-\sin\frac{\gamma}{2}\cos\frac{\beta-\delta}{2}-i\cos\frac{\gamma}{2}\sin\frac{\beta+\delta}{2}\\
            \sin\frac{\gamma}{2}\cos\frac{\beta-\delta}{2}-i\cos\frac{\gamma}{2}\sin\frac{\beta+\delta}{2}&\cos\frac{\gamma}{2}\cos\frac{\beta+\delta}{2}+i\sin\frac{\gamma}{2}\sin\frac{\beta-\delta}{2}
        \end{matrix}\right]\times\\
        \notag&e^{-i\alpha}\left[\begin{matrix}
            \cos\frac{\gamma}{2}\cos\frac{\beta+\delta}{2}-i\sin\frac{\gamma}{2}\sin\frac{\beta-\delta}{2}&-\sin\frac{\gamma}{2}\cos\frac{\beta-\delta}{2}-i\cos\frac{\gamma}{2}\sin\frac{\beta+\delta}{2}\\
            \sin\frac{\gamma}{2}\cos\frac{\beta-\delta}{2}-i\cos\frac{\gamma}{2}\sin\frac{\beta+\delta}{2}&\cos\frac{\gamma}{2}\cos\frac{\beta+\delta}{2}+i\sin\frac{\gamma}{2}\sin\frac{\beta-\delta}{2}
        \end{matrix}\right]^{\dagger}\\
        \notag=&\left[\begin{matrix}
            \cos\frac{\gamma}{2}\cos\frac{\beta+\delta}{2}-i\sin\frac{\gamma}{2}\sin\frac{\beta-\delta}{2}&-\sin\frac{\gamma}{2}\cos\frac{\beta-\delta}{2}-i\cos\frac{\gamma}{2}\sin\frac{\beta+\delta}{2}\\
            \sin\frac{\gamma}{2}\cos\frac{\beta-\delta}{2}-i\cos\frac{\gamma}{2}\sin\frac{\beta+\delta}{2}&\cos\frac{\gamma}{2}\cos\frac{\beta+\delta}{2}+i\sin\frac{\gamma}{2}\sin\frac{\beta-\delta}{2}
        \end{matrix}\right]\times\\
        \notag&\left[\begin{matrix}
            \cos\frac{\gamma}{2}\cos\frac{\beta+\delta}{2}+i\sin\frac{\gamma}{2}\sin\frac{\beta-\delta}{2}&\sin\frac{\gamma}{2}\cos\frac{\beta-\delta}{2}+i\cos\frac{\gamma}{2}\sin\frac{\beta+\delta}{2}\\
            -\sin\frac{\gamma}{2}\cos\frac{\beta-\delta}{2}+i\cos\frac{\gamma}{2}\sin\frac{\beta+\delta}{2}&\cos\frac{\gamma}{2}\cos\frac{\beta+\delta}{2}-i\sin\frac{\gamma}{2}\sin\frac{\beta-\delta}{2}
        \end{matrix}\right]\\
        \notag=&\left[\begin{smallmatrix}
            \left(\cos\frac{\gamma}{2}\cos\frac{\beta+\delta}{2}-i\sin\frac{\gamma}{2}\sin\frac{\beta-\delta}{2}\right)\left(\cos\frac{\gamma}{2}\cos\frac{\beta+\delta}{2}+i\sin\frac{\gamma}{2}\cos\frac{\beta-\delta}{2}\right)+\left(-\sin\frac{\gamma}{2}\cos\frac{\beta-\delta}{2}-i\cos\frac{\gamma}{2}\sin\frac{\beta+\delta}{2}\right)\left(-\sin\frac{\gamma}{2}\cos\frac{\beta-\delta}{2}+i\cos\frac{\gamma}{2}\sin\frac{\beta+\delta}{2}\right),\\
            \left(\cos\frac{\gamma}{2}\cos\frac{\beta+\delta}{2}-i\sin\frac{\gamma}{2}\sin\frac{\beta-\delta}{2}\right)\left(\sin\frac{\gamma}{2}\cos\frac{\beta-\delta}{2}+i\cos\frac{\gamma}{2}\sin\frac{\beta+\delta}{2}\right)+\left(-\sin\frac{\gamma}{2}\cos\frac{\beta-\delta}{2}-i\cos\frac{\gamma}{2}\sin\frac{\beta+\delta}{2}\right)\left(\cos\frac{\gamma}{2}\cos\frac{\beta+\delta}{2}-i\sin\frac{\gamma}{2}\sin\frac{\beta-\delta}{2}\right);\\
            \left(\sin\frac{\gamma}{2}\cos\frac{\beta-\delta}{2}-i\cos\frac{\gamma}{2}\sin\frac{\beta+\delta}{2}\right)\left(\cos\frac{\gamma}{2}\cos\frac{\beta+\delta}{2}+i\sin\frac{\gamma}{2}\sin\frac{\beta-\delta}{2}\right)+\left(\cos\frac{\gamma}{2}\cos\frac{\beta+\delta}{2}+i\sin\frac{\gamma}{2}\sin\frac{\beta-\delta}{2}\right)\left(-\sin\frac{\gamma}{2}\cos\frac{\beta-\delta}{2}+i\cos\frac{\gamma}{2}\sin\frac{\beta+\delta}{2}\right),\\
            \left(\sin\frac{\gamma}{2}\cos\frac{\beta-\delta}{2}-i\cos\frac{\gamma}{2}\sin\frac{\beta+\delta}{2}\right)\left(\sin\frac{\gamma}{2}\cos\frac{\beta-\delta}{2}+i\cos\frac{\gamma}{2}\sin\frac{\beta+\delta}{2}\right)+\left(\cos\frac{\gamma}{2}\cos\frac{\beta+\delta}{2}+i\sin\frac{\gamma}{2}\sin\frac{\beta-\delta}{2}\right)\left(\cos\frac{\gamma}{2}\cos\frac{\beta+\delta}{2}-i\sin\frac{\gamma}{2}\sin\frac{\beta-\delta}{2}\right)
        \end{smallmatrix}\right]\\
        \notag=&\left[\begin{smallmatrix}
            \cos^2\frac{\gamma}{2}\cos^2\frac{\beta+\delta}{2}+\sin^2\frac{\gamma}{2}\sin^2\frac{\beta-\delta}{2}+\sin^2\frac{\gamma}{2}\cos^2\frac{\beta-\delta}{2}+\cos^2\frac{\gamma}{2}\sin^2\frac{\beta+\delta}{2}&0\\
            0&\sin^2\frac{\gamma}{2}\cos^2\frac{\beta-\delta}{2}+\cos^2\frac{\gamma}{2}\sin^2\frac{\beta+\delta}{2}+\cos^2\frac{\gamma}{2}\cos^2\frac{\beta+\delta}{2}+\sin^2\frac{\gamma}{2}\sin^2\frac{\beta-\delta}{2}
        \end{smallmatrix}\right]\\
        =&\left[\begin{matrix}
            1&0\\
            0&1
        \end{matrix}\right]=I.
    \end{align}
    the $X-Y$ decomposition of rotations is also unitary.
    For arbitrary $a$, $b$, $c$, and $d$ such that $U$ is unitary, let
    \begin{align}
        a=&e^{i\alpha}\left(\cos\frac{\gamma}{2}\cos\frac{\beta+\delta}{2}-i\sin\frac{\gamma}{2}\sin\frac{\beta-\delta}{2}\right),\\
        b=&e^{i\alpha}\left(-\sin\frac{\gamma}{2}\cos\frac{\beta-\delta}{2}-i\cos\frac{\gamma}{2}\sin\frac{\beta+\delta}{2}\right),\\
        c=&e^{i\alpha}\left(\sin\frac{\gamma}{2}\cos\frac{\beta-\delta}{2}-i\cos\frac{\gamma}{2}\sin\frac{\beta+\delta}{2}\right),\\
        d=&e^{i\alpha}\left(\cos\frac{\gamma}{2}\cos\frac{\beta+\delta}{2}+i\sin\frac{\gamma}{2}\sin\frac{\beta-\delta}{2}\right),
    \end{align}
    i.e., set
    \begin{align}
        \alpha=&-\frac{i}{2}\ln(ad-bc),\\
        \gamma=&\arccos\frac{a^2-b^2-c^2+b^2}{2(ad-bc)},\\
        \beta=&i\left(\arctan\frac{b+c}{a+d}-\arctan\frac{a-d}{b-c}\right),\\
        \delta=&i\left(\arctan\frac{b+c}{a+d}+\arctan\frac{a-d}{b-c}\right).
    \end{align}
    Then
    \begin{align}
        U=e^{i\alpha}R_z(\beta)R_x(\gamma)R_z(\delta),
    \end{align}
    which is a $X-Y$ decomposition of rotations analogous to Theorem 4.1.
\end{sol}

\begin{exe}
    Suppose $\hat{m}$ and $\hat{n}$ are non-parallel real unit vectors in three dimensions. Use Theorem 4.1 to show that an arbitrary single qubit unitary $U$ may be written
    \begin{align}
        U=e^{i\alpha}R_{\hat{n}}(\beta)R_{\hat{m}}(\gamma)R_{\hat{n}}(\delta),
    \end{align}
    for appropriate choices of $\alpha$, $\beta$, $\gamma$ and $\delta$.
\end{exe}
\begin{pf}
    There is something wrong with this Exercise. Imagine that $\hat{m}$ and $\hat{n}$ are almost parallel but not parallel. In this case, we can not rotate a Bloch vector from paralleling with $\hat{n}$ to anti-paralleling with $\hat{n}$ by first rotating the vector about $\hat{n}$ by some angle $\delta$, then about $\hat{m}$ by some angle $\gamma$, and finally about $\hat{n}$ by some angle $\beta$, with a phase $\alpha$ added. However, this rotation can be represented by unitary operator $U=X$. Therefore, some single qubit unitary $U$ may not be written
    \begin{align}
        U=e^{i\alpha}R_{\hat{n}}(\beta)R_{\hat{m}}(\gamma)R_{\hat{n}}(\delta),
    \end{align}
    with any choices of $\alpha$, $\beta$, $\gamma$ and $\delta$.
\end{pf}

\begin{exe}
    Give $A$, $B$, $C$, and $\alpha$ for the Hadamard gate.
\end{exe}
\begin{sol}
    The Hadamard gate can be written as
    \begin{align}
        H\equiv\frac{1}{\sqrt{2}}\left[\begin{matrix}
            1&1\\
            1&-1
        \end{matrix}\right]=e^{i\alpha}R_z(\beta)R_y(\gamma)R_z(\beta)=\left[\begin{matrix}
            e^{i(\alpha-\beta/2-\delta/2)}\cos\frac{\gamma}{2}&-e^{i(\alpha-\beta/2+\delta/2)}\sin\frac{\gamma}{2}\\
            e^{i(\alpha+\beta/2-\delta/2)}\sin\frac{\gamma}{2}&e^{i(\alpha+\beta/2+\delta/2)}\cos\frac{\gamma}{2}
        \end{matrix}\right],
    \end{align}
    i.e.,
    \begin{align}
        e^{i(\alpha-\beta/2-\delta/2)}\cos\frac{\gamma}{2}=&\frac{1}{\sqrt{2}},\\
        -e^{i(\alpha-\beta/2+\delta/2)}\sin\frac{\gamma}{2}=&\frac{1}{\sqrt{2}},\\
        e^{i(\alpha+\beta/2-\delta/2)}\sin\frac{\gamma}{2}=&\frac{1}{\sqrt{2}},\\
        e^{i(\alpha+\beta/2+\delta/2)}\cos\frac{\gamma}{2}=&-\frac{1}{\sqrt{2}},
    \end{align}
    so
    \begin{align}
        \alpha=&\frac{\pi}{2},\\
        \beta=&0,\\
        \delta=&\pi,\\
        \gamma=&\frac{\pi}{2}.
    \end{align}
    In this way, for the Hadamard gate,
    \begin{align}
        H=e^{i\alpha}AXBXC,
    \end{align}
    where
    \begin{align}
        \alpha=&\frac{\pi}{2},\\
        A=&R_z(\beta)R_y(\gamma/2)=\left[\begin{matrix}
            e^{-i\beta/2}&0\\
            0&e^{i\beta/2}
        \end{matrix}\right]\left[\begin{matrix}
            \cos\frac{\gamma}{4}&-\sin\frac{\gamma}{4}\\
            \sin\frac{\gamma}{4}&\cos\frac{\gamma}{4}
        \end{matrix}\right]=\left[\begin{matrix}
            e^{-i\beta/2}\cos\frac{\gamma}{4}&-e^{-i\beta/2}\sin\frac{\gamma}{4}\\
            e^{i\beta/2}\sin\frac{\gamma}{4}&e^{i\beta/2}\cos\frac{\gamma}{4}
        \end{matrix}\right]=\left[\begin{matrix}
            \cos\frac{\pi}{8}&-\sin\frac{\pi}{8}\\
            \sin\frac{\pi}{8}&\cos\frac{\pi}{8}
        \end{matrix}\right],\\
        \notag B=&R_y(-\gamma/2)R_z(-(\delta+\beta)/2)=\left[\begin{matrix}
            \cos\left(-\frac{\gamma}{4}\right)&-\sin\left(-\frac{\gamma}{4}\right)\\
            \sin\left(-\frac{\gamma}{4}\right)&\cos\left(-\frac{\gamma}{4}\right)
        \end{matrix}\right]\left[\begin{matrix}
            e^{i(\delta+\beta)/4}&0\\
            0&e^{-i(\delta+\beta)/4}
        \end{matrix}\right]=\left[\begin{matrix}
            e^{i(\delta+\beta)/4}\cos\frac{\gamma}{4}&e^{-i(\delta+\beta)/4}\sin\frac{\gamma}{4}\\
            -e^{i(\delta+\beta)/4}\sin\frac{\gamma}{4}&e^{-i(\delta+\beta)/4}\cos\frac{\gamma}{4}
        \end{matrix}\right]\\
        =&\left[\begin{matrix}
            e^{i\pi/4}\cos\frac{\pi}{8}&e^{-i\pi/4}\sin\frac{\pi}{8}\\
            -e^{i\pi/4}\sin\frac{\pi}{8}&e^{-i\pi/4}\cos\frac{\pi}{8}
        \end{matrix}\right],\\
        C=&R_z((\delta-\beta)/2)=\left[\begin{matrix}
            e^{-i(\delta-\beta)/4}&0\\
            0&e^{i(\delta-\beta)/4}
        \end{matrix}\right]=\left[\begin{matrix}
            e^{-i\pi/4}&0\\
            0&e^{i\pi/4}
        \end{matrix}\right]
    \end{align}
\end{sol}

\begin{exe}[Circuit identities]
    It is useful to be able to simplify circuit by inspection, using well-known identities. Prove the following three identities:
    \begin{align}
        HXH=Z;\quad HYH=-Y,\quad HZH=X.
    \end{align}
\end{exe}
\begin{pf}
    \begin{align}
        HXH=&\frac{1}{\sqrt{2}}\left[\begin{matrix}
            1&1\\
            1&-1
        \end{matrix}\right]\left[\begin{matrix}
            0&1\\
            1&0
        \end{matrix}\right]\frac{1}{\sqrt{2}}\left[\begin{matrix}
            1&1\\
            1&-1
        \end{matrix}\right]=\left[\begin{matrix}
            1&0\\
            0&-1
        \end{matrix}\right]=Z,\\
        HYH=&\frac{1}{\sqrt{2}}\left[\begin{matrix}
            1&1\\
            1&-1
        \end{matrix}\right]\left[\begin{matrix}
            0&-i\\
            i&0
        \end{matrix}\right]\frac{1}{\sqrt{2}}\left[\begin{matrix}
            1&1\\
            1&-1
        \end{matrix}\right]=\left[\begin{matrix}
            0&i\\
            -i&0
        \end{matrix}\right]=-Y,\\
        HZH=&\frac{1}{\sqrt{2}}\left[\begin{matrix}
            1&1\\
            1&-1
        \end{matrix}\right]\left[\begin{matrix}
            1&0\\
            0&-1
        \end{matrix}\right]\frac{1}{\sqrt{2}}\left[\begin{matrix}
            1&1\\
            1&-1
        \end{matrix}\right]=\left[\begin{matrix}
            0&1\\
            1&0
        \end{matrix}\right]=X.
    \end{align}
\end{pf}

\ifx\allfiles\undefined
\end{document}
\fi