% !Tex program = pdflatex
% Chapter 5 - The quantum Fourier transform and its applications
\ifx\allfiles\undefined
\documentclass[en]{sol-man}
\begin{document}
\fi
\chapter{The quantum Fourier transform and its applications}

\section{The quantum Fourier transform}

\begin{exe}
    Give a direct proof that the linear transform defined by Equation (5.2)\footnote{\label{Equ-5.2}$\lvert j\rangle\longrightarrow\frac{1}{\sqrt{N}}\sum_{k=0}^{N-1}e^{2\pi ijk/N}\lvert k\rangle$.} is unitary.
\end{exe}
\begin{pf}
    Set the matrix of the linear transform defined by Equation (5.2) as $\hat{U}$,
    \begin{align}
        \hat{U}\lvert j\rangle=\frac{1}{\sqrt{N}}\sum_{k=0}^{N-1}e^{2\pi ijk/N}\lvert k\rangle.
    \end{align}
    Since for any $\lvert j\rangle$
    \begin{align}
        \notag\langle j\rvert\hat{U}^{\dagger}\hat{U}\lvert j\rangle=&\left(\frac{1}{\sqrt{N}}\sum_{l=0}^{N-1}e^{2\pi ijl/N}\lvert l\rangle\right)^{\dagger}\left(\frac{1}{\sqrt{N}}\sum_{k=0}^{N-1}e^{2\pi ijk/N}\lvert k\rangle\right)=\frac{1}{N}\sum_{k,l=0}^{N-1}e^{2\pi ij(k-l)/N}\langle l\vert k\rangle=\frac{1}{N}\sum_{k,l=0}^{N-1}e^{2\pi ij(k-l)/N}\delta_{lk}\\
        =&\frac{1}{N}\sum_{k=0}^{N-1}1=1=\langle j\vert j\rangle,
    \end{align}
    \begin{align}
        \Longrightarrow\hat{U}^{\dagger}\hat{U}=I,
    \end{align}
    i.e., the linear transform defined by Equation (5.2)\footref{Equ-5.2} is unitary.
\end{pf}

\begin{exe}
    Explicitly compute the Fourier transform of the $n$ qubit state $\lvert 00\dots 0\rangle$.
\end{exe}
\begin{sol}
    The Fourier transform of the $n$ qubit state $\lvert 00\dots 0\rangle$ is
    \begin{align}
        \lvert 00\dots 0\rangle=\lvert 0\rangle\longrightarrow\frac{1}{2^{n/2}}\sum_{k=0}^{2^n-1}\lvert k\rangle=\frac{1}{2^{n/2}}=\left[\frac{1}{\sqrt{2}}(\lvert 0\rangle+\lvert 1\rangle)\right]^n.
    \end{align}
\end{sol}

\begin{exe}[Classical fast Fourier transform]
    Suppose we wish to perform a Fourier transform of a vector containing $2^n$ complex numbers on a classical computer. Verify that the straightforward method for performing the Fourier transform, based upon direct evaluation of Equation (5.1)\footnote{\label{Equ-5.1}$y_k\equiv\frac{1}{\sqrt{N}}\sum_{j=0}^{N-1}x_je^{2\pi ijk/N}$.} requires $\Theta(2^{2n})$ elementary arithmetic operations. Find a method for reducing this to $\Theta(n2^n)$ operations, based upon Equation (5.4)\footnote{\label{Equ-5.4}$\lvert j_1,\cdots,j_n\rangle\rightarrow\frac{\left(\lvert 0\rangle+e^{2\pi i0.j_n}\lvert 1\rangle\right)\left(\lvert 0\rangle+e^{2\pi i0.j_{n-1}j_n}\lvert 1\rangle\right)\cdots\left(\lvert 0\rangle+e^{2\pi i0.j_1j_2\cdots j_n}\lvert 1\rangle\right)}{2^{n/2}}$}.
\end{exe}
\begin{sol}
    According to Equation (5.1)\footref{Equ-5.1}, every element $y_k$ in vector $\lvert y\rangle$ takes $2^n$ elementary arithmetic operations. Since there are $2^n$ elements in vector $\lvert y\rangle$, the straightforward method for performing the Fourier transform, based upon direct evaluation of Equation (5.1)\footref{Equ-5.1} requires $\Theta(2^{2n})$ elementary arithmetic operations.

    Based upon Equation (5.1)\footref{Equ-5.1}. For every element $y_k$ in vector $\lvert y\rangle$,
    \begin{align}
        \notag y_k=&\frac{1}{2^{n/2}}\sum_{j=0}^{2^n-1}x_je^{2\pi ijk/N}=\sum_{j=0}^{2^{n-1}-1}x_{2j}e^{2\pi i(2j)k/N}+\sum_{j=0}^{2^{n-1}-1}x_{2j+1}e^{2\pi i(2j+1)k/N}\\
        =&\sum_{j=0}^{2^{n-1}-1}x_{2j}e^{2\pi i(2j)k/N}+e^{2\pi ik/N}\sum_{j=0}^{2^{n-1}-1}x_{2j+1}e^{2\pi i(2j)k/N}.
    \end{align}
    Define $T(\text{A})$ as the elementary arithmetic operation number needed (computational complexity) for algorithm A. Here,
    \begin{align}
        \notag&T(\text{Fourier transform of a vector containing $2^n$ complex numbers})\\
        \notag=&T(\text{calculate $\sum_{j=0}^{2^{n-1}-1}x_{2j}e^{2\pi i(2j)k/N}$ for }k=1,2,\cdots,2^n)+T(\text{calculate $\sum_{j=0}^{2^{n-1}-1}x_{2j+1}e^{2\pi i(2j)k/N}$ for }k=1,2,\cdots,2^n)+2^n\\
        \notag=&2T(\text{Fourier transform of a vector containing $2^{n-1}$ complex numbers})+2^n.
    \end{align}
    In this way,
    \begin{align}
        T(\text{Fourier transform of a vector containing $2^n$ complex numbers})=O(n2^n),
    \end{align}
    i.e., we reduce Fourier transform of a vector containing $2^n$ complex numbers to $\Theta(n2^n)$ elementary arithmetic operations.
\end{sol}

\ifx\allfiles\undefined
\end{document}
\fi