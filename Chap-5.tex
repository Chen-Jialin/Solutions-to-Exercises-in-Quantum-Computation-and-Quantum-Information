% !Tex program = pdflatex
% Chapter 5 - The quantum Fourier transform and its applications
\ifx\allfiles\undefined
\documentclass[en]{sol-man}
\begin{document}
\fi
\chapter{The quantum Fourier transform and its applications}

\section{The quantum Fourier transform}

\begin{exe}
    Give a direct proof that the linear transform defined by Equation (5.2)\footnote{\label{Equ-5.2}$\lvert j\rangle\longrightarrow\frac{1}{\sqrt{N}}\sum_{k=0}^{N-1}e^{2\pi ijk/N}\lvert k\rangle$.} is unitary.
\end{exe}
\begin{pf}
    Set the matrix of the linear transform defined by Equation (5.2) as $\hat{U}$,
    \begin{align}
        \hat{U}\lvert j\rangle=\frac{1}{\sqrt{N}}\sum_{k=0}^{N-1}e^{2\pi ijk/N}\lvert k\rangle.
    \end{align}
    Since for any $\lvert j\rangle$
    \begin{align}
        \notag\langle j\rvert\hat{U}^{\dagger}\hat{U}\lvert j\rangle=&\left(\frac{1}{\sqrt{N}}\sum_{l=0}^{N-1}e^{2\pi ijl/N}\lvert l\rangle\right)^{\dagger}\left(\frac{1}{\sqrt{N}}\sum_{k=0}^{N-1}e^{2\pi ijk/N}\lvert k\rangle\right)=\frac{1}{N}\sum_{k,l=0}^{N-1}e^{2\pi ij(k-l)/N}\langle l\vert k\rangle=\frac{1}{N}\sum_{k,l=0}^{N-1}e^{2\pi ij(k-l)/N}\delta_{lk}\\
        =&\frac{1}{N}\sum_{k=0}^{N-1}1=1=\langle j\vert j\rangle,
    \end{align}
    \begin{align}
        \Longrightarrow\hat{U}^{\dagger}\hat{U}=I,
    \end{align}
    i.e., the linear transform defined by Equation (5.2)\footref{Equ-5.2} is unitary.
\end{pf}

\begin{exe}
    Explicitly compute the Fourier transform of the $n$ qubit state $\lvert 00\dots 0\rangle$.
\end{exe}
\begin{sol}
    The Fourier transform of the $n$ qubit state $\lvert 00\dots 0\rangle$ is
    \begin{align}
        \lvert 00\dots 0\rangle=\lvert 0\rangle\longrightarrow\frac{1}{2^{n/2}}\sum_{k=0}^{2^n-1}\lvert k\rangle=\frac{1}{2^{n/2}}=\left[\frac{1}{\sqrt{2}}(\lvert 0\rangle+\lvert 1\rangle)\right]^n.
    \end{align}
\end{sol}

\begin{exe}[Classical fast Fourier transform]
    Suppose we wish to perform a Fourier transform of a vector containing $2^n$ complex numbers on a classical computer. Verify that the straightforward method for performing the Fourier transform, based upon direct evaluation of Equation (5.1)\footnote{\label{Equ-5.1}$y_k\equiv\frac{1}{\sqrt{N}}\sum_{j=0}^{N-1}x_je^{2\pi ijk/N}$.} requires $\Theta(2^{2n})$ elementary arithmetic operations. Find a method for reducing this to $\Theta(n2^n)$ operations, based upon Equation (5.4)\footnote{\label{Equ-5.4}$\lvert j_1,\cdots,j_n\rangle\rightarrow\frac{\left(\lvert 0\rangle+e^{2\pi i0.j_n}\lvert 1\rangle\right)\left(\lvert 0\rangle+e^{2\pi i0.j_{n-1}j_n}\lvert 1\rangle\right)\cdots\left(\lvert 0\rangle+e^{2\pi i0.j_1j_2\cdots j_n}\lvert 1\rangle\right)}{2^{n/2}}$}.
\end{exe}
\begin{sol}
    According to Equation (5.1)\footref{Equ-5.1}, every element $y_k$ in vector $\lvert y\rangle$ takes $2^n$ elementary arithmetic operations. Since there are $2^n$ elements in vector $\lvert y\rangle$, the straightforward method for performing the Fourier transform, based upon direct evaluation of Equation (5.1)\footref{Equ-5.1} requires $\Theta(2^{2n})$ elementary arithmetic operations.

    Based upon Equation (5.1)\footref{Equ-5.1}. For every element $y_k$ in vector $\lvert y\rangle$,
    \begin{align}
        \notag y_k=&\frac{1}{2^{n/2}}\sum_{j=0}^{2^n-1}x_je^{2\pi ijk/N}=\sum_{j=0}^{2^{n-1}-1}x_{2j}e^{2\pi i(2j)k/N}+\sum_{j=0}^{2^{n-1}-1}x_{2j+1}e^{2\pi i(2j+1)k/N}\\
        =&\sum_{j=0}^{2^{n-1}-1}x_{2j}e^{2\pi i(2j)k/N}+e^{2\pi ik/N}\sum_{j=0}^{2^{n-1}-1}x_{2j+1}e^{2\pi i(2j)k/N}.
    \end{align}
    Define $T(\text{A})$ as the elementary arithmetic operation number needed (computational complexity) for algorithm A. Here,
    \begin{align}
        \notag&T(\text{Fourier transform of a vector containing $2^n$ complex numbers})\\
        \notag=&T(\text{calculate $\sum_{j=0}^{2^{n-1}-1}x_{2j}e^{2\pi i(2j)k/N}$ for }k=1,2,\cdots,2^n)+T(\text{calculate $\sum_{j=0}^{2^{n-1}-1}x_{2j+1}e^{2\pi i(2j)k/N}$ for }k=1,2,\cdots,2^n)+2^n\\
        \notag=&2T(\text{Fourier transform of a vector containing $2^{n-1}$ complex numbers})+2^n.
    \end{align}
    In this way,
    \begin{align}
        T(\text{Fourier transform of a vector containing $2^n$ complex numbers})=O(n2^n),
    \end{align}
    i.e., we reduce Fourier transform of a vector containing $2^n$ complex numbers to $\Theta(n2^n)$ elementary arithmetic operations.
\end{sol}

\begin{exe}
    Give a decomposition of the controlled-$R_k$ gate into single qubit and CNOT gates.
\end{exe}
\begin{sol}
    The $R_k$ gate can be written as
    \begin{align}
        R_k=\begin{bmatrix}
            1&0\\
            0&e^{2\pi i/2^k}
        \end{bmatrix}=e^{i\alpha}\begin{bmatrix}
            e^{i(\alpha-\beta/2-\delta/2)}\cos\frac{\gamma}{2}&-e^{i(\alpha-\beta/2+\delta/2)}\sin\frac{\gamma}{2}\\
            e^{i(\alpha+\beta/2-\delta/2)}\sin\frac{\gamma}{2}&e^{i(\alpha+\beta/2+\delta/2)}\cos\frac{\gamma}{2}
        \end{bmatrix},
    \end{align}
    i.e.,
    \begin{align}
        e^{i(\alpha-\beta/2-\delta/2)}\cos\frac{\gamma}{2}=&1,\\
        -e^{i(\alpha-\beta/2+\delta/2)}\sin\frac{\gamma}{2}=&0,\\
        e^{i(\alpha+\beta/2-\delta/2)}\sin\frac{\gamma}{2}=&0,\\
        e^{i(\alpha+\beta/2+\delta/2)}\cos\frac{\gamma}{2}=&e^{2\pi i/2^k},
    \end{align}
    so we can choose
    \begin{align}
        \alpha=&\pi/2^k,\\
        \beta=&2\pi/2^k,\\
        \delta=&0,\\
        \gamma=&0.
    \end{align}
    In this way, $R_k$ gate can be decomposed as
    \begin{align}
        R_k=e^{i\alpha}AXBXC,
    \end{align}
    where
    \begin{align}
        \alpha=&\frac{\pi}{2^k},\\
        A=&R_z(\beta)R_y(\gamma/2)=\begin{bmatrix}
            e^{-i\beta/2}&0\\
            0&e^{i\beta/2}
        \end{bmatrix}\begin{bmatrix}
            \cos\frac{\gamma}{4}&-\sin\frac{\gamma}{4}\\
            \sin\frac{\gamma}{4}&\cos\frac{\gamma}{4}
        \end{bmatrix}=\begin{bmatrix}
            e^{-i\beta/2}\cos\frac{\gamma}{4}&-e^{-i\beta/2}\cos\frac{\gamma}{4}\\
            e^{i\beta/2}\sin\frac{\gamma}{4}&e^{i\beta/2}\cos\frac{\gamma}{4}
        \end{bmatrix}=\begin{bmatrix}
            e^{-i\pi/2^k}&0\\
            0&e^{i\pi/2^k}
        \end{bmatrix},\\
        \notag B=&R_y(-\gamma/2)R_z(-(\delta+\beta)/2)=\begin{bmatrix}
            \cos\left(-\frac{\gamma}{4}\right)&-\sin\left(-\frac{\gamma}{4}\right)\\
            \sin\left(-\frac{\gamma}{4}\right)&\cos\left(-\frac{\gamma}{4}\right)
        \end{bmatrix}\begin{bmatrix}
            e^{i(\delta+\beta)/4}&0\\
            0&e^{-i(\delta+\beta)/4}
        \end{bmatrix}=\begin{bmatrix}
            e^{i(\delta+\beta)/4}\cos\frac{\gamma}{4}&e^{-i(\delta+\beta)/4}\sin\frac{\gamma}{4}\\
            -e^{i(\delta+\beta)/4}\sin\frac{\gamma}{4}&e^{-i(\delta+\beta)/4}\cos\frac{\gamma}{4}
        \end{bmatrix}\\
        =&\begin{bmatrix}
            e^{i\pi/2^{k+1}}&0\\
            0&e^{-i\pi/2^{k+1}}
        \end{bmatrix},\\
        C=&R_z((\delta-\beta)/2)=\begin{bmatrix}
            e^{-i(\delta-\beta)/4}&0\\
            0&e^{i(\delta-\beta)/4}
        \end{bmatrix}=\begin{bmatrix}
            e^{-i\pi/2^{k+1}}&0\\
            0&e^{i\pi/2^{k+1}}
        \end{bmatrix}.
    \end{align}
    Using the equation in Figure 4.6\footref{Fig-4.6-controlled-U-construction}, the decomposition of the controlled-$R_k$ gate into single qubit and CNOT gates is
    \begin{center}
        \begin{quantikz}
            \qw & \ctrl{1} & \qw\\
            \qw & \gate{R_k} & \qw
        \end{quantikz}$=$\begin{tikzcd}[ampersand replacement=\&]
            \qw \& \qw \& \ctrl{1} \& \qw \& \ctrl{1} \& \gate{\begin{bmatrix}1&0\\0&e^{i\alpha}\end{bmatrix}} \& \qw\\
            \qw \& \gate{C} \& \targ{} \& \gate{B} \& \targ{} \& \gate{A} \& \qw
        \end{tikzcd}
    \end{center}
\end{sol}

\ifx\allfiles\undefined
\end{document}
\fi