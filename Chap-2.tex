% !Tex program = pdflatex
% Chapter 2 - Introduction to quantum mechanics
\ifx\allfiles\undefined
\documentclass[en]{sol-man}
\begin{document}
\fi
\chapter{Introduction to quantum mechanics}

\section{Linear algebra}

\begin{exe}[Linear dependence: example]
    Show that $(1,-1)$, $(1,2)$ and $(2,1)$ are Linearly dependent.
\end{exe}
\begin{sol}
    Since
    \begin{align}
        (1,-1)+(1,2)-(2,1)=0,
    \end{align}
    these three vectors are linearly dependent.
\end{sol}

\begin{exe}[Matrix representations: example]
    Suppose $V$ is a vector space with basis vectors $\lvert 0\rangle$ and $\lvert 1\rangle$, and $A$ is a linear operator from $V$ to $V$ such that $A\lvert 0\rangle=\lvert 1\rangle$ and $A\lvert 1\rangle=\lvert 0\rangle$. Give a matrix representation for $A$, with respect to the input basis $\lvert 0\rangle$, $\lvert 1\rangle$, and the output basis $\lvert 0\rangle$ and $\lvert 1\rangle$. Find input and output bases which give rise to a different matrix representation of $A$.
\end{exe}
\begin{sol}
    The matrix representation for $A$ with respect to the input basis $\lvert 0\rangle$, $\lvert 1\rangle$ and the output basis $\lvert 0\rangle$ and $\lvert 1\rangle$ is
    \begin{align}
        \left[\begin{matrix}
            0&1\\
            1&0
        \end{matrix}\right].
    \end{align}

    Keep $\lvert 0\rangle$ and $\lvert 1\rangle$ as the input basis and choose $\lvert+\rangle=(\lvert 0\rangle+\lvert 1\rangle)/\sqrt{2}$ and $\lvert-\rangle=(\lvert 0\rangle-\lvert 1\rangle)/\sqrt{2}$ as the output basis, then $A$ can be regarded as a linear operator from $V$ to $V$ such that $A\lvert 0\rangle=(\lvert+\rangle-\lvert-\rangle)/\sqrt{2}$ and $A\lvert 1\rangle=(\lvert 0\rangle+\lvert 1\rangle)/\sqrt{2}$. In this way, the matrix representation for $A$ is
    \begin{align}
        \frac{1}{\sqrt{2}}\left[\begin{matrix}
            1&-1\\
            1&1
        \end{matrix}\right].
    \end{align}
\end{sol}

\ifx\allfiles\undefined
\end{document}
\fi